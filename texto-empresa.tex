
% ----------------------------------------------------------
% Introdução (exemplo de capítulo sem numeração, mas presente no Sumário)
% ----------------------------------------------------------
\chapter[Introdução]{Introdução à empresa}

\section{Apresentação da empresa}

EY também conhecida como Ernst & Young Global Limited é uma empresa multinacional de serviços profissionais e é considerada junto com a Deloitte, KPMG e PricewaterhouseCoopers uma das maiores empresas de contabilidade do mundo. 

\section{Histórico da empresa}

A EY é uma empresa que nasceu da fusão em 1989 de duas empresas britânicas de contabilidade, sendo elas a Ernst & Whinney e Arthur Young & Co. Desde 2013 a empresa passou a se chamar EY e mudou seu slogan para Building a Better Working World. Hoje possuí mais de 700 escritórios que estão espalhados em mais de 150 países com 260 mil funcionários; No Brasil são 14 escritórios em 12 cidades com 5 mil funcionários. 
Atualmente a EY possui quatro linhas de serviços principais: Consultoria, Auditoria, Impostos e Transações. Com esses pilares eles buscam ajudar organizações privadas ou publicas a capitalizarem mais oportunidades, não importando se são empresas pequenas ou grandes empresas. 

